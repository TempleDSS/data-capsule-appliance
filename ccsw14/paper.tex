\documentclass{acm_proc_article-sp}

\usepackage{epstopdf}

\begin{document}

\title{Securing Remote Access to Sensitive Data}

%\numberofauthors{2} 
%
%\author{
%\alignauthor Alexander Crowell \\
%       \affaddr{Computer Science and Engineering Division}\\
%       \affaddr{University of Mighigan}\\
%       \affaddr{Ann Arbor, MI, USA}\\
%       \email{crowella@umich.edu}
%\alignauthor Atul Prakash \\
%       \affaddr{Computer Science and Engineering Division}\\
%       \affaddr{University of Mighigan}\\
%       \affaddr{Ann Arbor, MI, USA}\\
%       \email{aprakash@umich.edu}
%}

\maketitle
\begin{abstract}

As we enter an era of ``big data", where analysis of large-scale data is
revealing important insights into a variety of fields, there is an ever greater
demand for access to new data wherein the potential for new insights may lie.
However, in many cases, this need for access conflicts with the desire to
protect data privacy.  Indeed, many types of important data present just such a
dilemma, including copyrighted data, personal information of individuals, and
data containing state or corporate secrets.

In this paper, we propose Data Capsules, a system that is designed to enable
access to sensitive data for analysis by trusted remote users, while
maintaining reasonable guarantees of data security.  Data Capsules uses
virtualization to provide remote users with a privileged, but secure
environment into which they can bring arbitrary, and even potentially
malicious, software or data in order to analyze sensitive data, while
minimizing the available channels for data leaks.  Our early implementation
realizes much of this protection, providing a basic framework for secure analysis of data that addresses many aspects of network, storage, and covert channel security.

\end{abstract}

\category{D.4.6}{Operating Systems}{Security and Protection}
\category{H.3.4}{Systems and Software}{Distributed Systems}[cloud computing, data capsules]

\keywords{data capsules, data privacy}

\section{Introduction}

In the age of the Internet, new data is being created at incredible rates
\cite{digital-universe}.  Along with these waves of new data, new tools have
in turn been created to help analyze and gain insight from it. Unfortunately,
access to data is not always a simple matter.  Some types of data may be
sensitive, including data protected by copyright, data representing the personal
information of individuals, as well as any data that may contain secrets that
its owner does not wish to have revealed, for example corporate or state secrets.

One good example of a system for access to sensitive data is the data for the
United States Census.  The federal government established the Census Research
Data Centers to enable researchers access to unpublished Census data for the
purpose of conducting research that could provide guidance on public policy.

\section{Related Work}
Donec ornare iaculis nisl, sit amet luctus metus varius ac. Nam eget tincidunt
magna. Sed convallis pulvinar eros sed consequat. Pellentesque posuere orci id
leo molestie, ac rutrum dolor pretium. In orci lectus, sollicitudin nec tellus
non, dictum egestas ante. Ut venenatis libero risus, sed auctor lorem
scelerisque at. Suspendisse porta elementum magna, id ornare erat placerat at.
In tincidunt dolor porta justo rhoncus elementum.

\section{Background}
Nullam viverra turpis nisi, id consequat urna pulvinar eu. Praesent ac pulvinar
nisl. Integer vel est a libero molestie mollis id vel tortor. Morbi ante odio,
cursus et volutpat a, pharetra at erat. Pellentesque habitant morbi tristique
senectus et netus et malesuada fames ac turpis egestas. Curabitur dignissim,
purus at auctor laoreet, magna mi auctor nulla, et mollis eros nisl vel lectus.
Integer bibendum purus ac adipiscing bibendum.

\section{Design and Implementation}
Suspendisse potenti. Donec ornare metus sem, vel semper arcu fermentum in. Nunc
vehicula eros turpis, ut rhoncus nunc sodales pulvinar. Duis sed est interdum,
semper mi a, cursus sapien. Nulla elementum congue purus, eu faucibus nibh
congue et. Ut diam metus, euismod ac quam in, gravida dictum turpis. Vivamus
hendrerit facilisis tempor. Curabitur hendrerit orci ac libero vestibulum, sit
amet dictum enim viverra. Nulla rutrum dolor ligula, nec rutrum massa suscipit
volutpat. Aliquam id mi at velit aliquet imperdiet at nec mi. Integer in massa
nec augue dapibus ullamcorper. Aliquam nec sollicitudin elit.

\section{Discussion and Future Work}
In eu gravida ligula, sit amet tincidunt nisi. Interdum et malesuada fames ac
ante ipsum primis in faucibus.  Vestibulum venenatis non sem vitae hendrerit.
Cras accumsan vel mauris ut vestibulum. Pellentesque auctor quam ut nisl
commodo malesuada. Phasellus adipiscing placerat tortor, in convallis massa
cursus eu. Vestibulum ullamcorper lorem vel nunc dictum, elementum pretium
nulla tincidunt.

\section{Conclusions}

Interdum et malesuada fames ac ante ipsum primis in faucibus. Nulla facilisi.
Etiam viverra condimentum accumsan. Mauris sed dolor sed mi dignissim
vulputate. Duis \cite{clark:pct} pharetra purus purus, eu ultricies ligula dignissim vel. Sed
nec mauris a turpis luctus placerat id ut ipsum. Fusce quis tincidunt dolor.
Nunc mattis ornare sem, vel consectetur ipsum iaculis ac. Aenean nulla lacus,
iaculis sed tempor nec, euismod suscipit dui.

% References
\bibliographystyle{abbrv}
\bibliography{paper}

%\balancecolumns
\end{document}
